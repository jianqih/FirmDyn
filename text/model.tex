\documentclass[11pt,a4paper]{article}

\usepackage{graphicx,hyperref,amsmath,natbib,bm,url}
\usepackage[utf8]{inputenc} % include umlaute

% -------- Define new color ------------------------------
\usepackage{color}
\definecolor{mygrey}{rgb}{0.85,0.85,0.85} % make grey for the table
%   }}
\definecolor{darkblue}{rgb}{0.055,0.094,0.7}
\definecolor{darkred}{rgb}{0.6,0,0}

% --------------------------------------------------------
\usepackage{microtype,todonotes}
\usepackage[australian]{babel} % change date to to European format
\usepackage[a4paper,text={14.5cm, 25.2cm},centering]{geometry} % Change page size
\setlength{\parskip}{1.2ex} % show new paragraphs with a space between lines
\setlength{\parindent}{0em} % get rid of indentation for new paragraph
\clubpenalty = 10000 % prevent "orphans" 
\widowpenalty = 10000 % prevent "widows"

\hypersetup{pdfpagemode=UseNone} % un-comment this if you don't want to show bookmarks open when opening the pdf

\usepackage{mathpazo} % change font to something close to Palatino

\usepackage{longtable, booktabs, tabularx} % for nice tables
\usepackage{caption,fixltx2e}  % for nice tables
\usepackage[flushleft]{threeparttable}  % for nice tables

\newcommand*{\myalign}[2]{\multicolumn{1}{#1}{#2}} % define new command so that we can change alignment by hand in rows
 
% -------- For use in the bibliography -------------------
\usepackage{multicol}
\usepackage{etoolbox}
\usepackage{relsize}
\setlength{\columnsep}{1cm} % change column separation for multi-columns
% \patchcmd{\thebibliography}
%   {\list}
%   {\begin{multicols}{2}\smaller\list}
%   {}
%   {}
% \appto{\endthebibliography}{\end{multicols}}
% % --------------------------------------------------------


\usepackage[onehalfspacing]{setspace} % Line spacing

\usepackage[marginal]{footmisc} % footnotes not indented
\setlength{\footnotemargin}{0.2cm} % set margin for footnotes (so that the number doesn't stick out) 

\usepackage{pdflscape} % for landscape figures

\usepackage[capposition=top]{floatrow} % For notes below figure

% -------- Use the following two lines for making lines grey in tables -------------------
\usepackage{color, colortbl}
\usepackage{multirow}
% ----------------------------------------------------------------------------------------
\newtheorem{defn}{Definition}[section]
\newtheorem{reg}{Rule}[section]
\newtheorem{exer}{Exercise}[section]
\newtheorem{note}{Note}[section]
\newtheorem{expl}{Example}[section]
\newtheorem{pro}{properties}[section]
\newtheorem{asm}{Assumption}
\newtheorem{them}{Theorem}[section]
\newtheorem{rmk}{Remark}[section]
\newtheorem{coro}{Corollary}[section]
\newtheorem{lem}{Lemma}[section]
\newtheorem{pros}{Proposition}[section]


\usepackage{dsfont}
% \tcbset{highlight math style={enhanced,
%     colframe=red!60!black,colback=yellow!50!white,arc=4pt,boxrule=1pt,
\hypersetup{colorlinks=true,           % put a box around links
  linkbordercolor = {1 0 0}, % the box will be red
  pdfborder = {1 0 0},       % 
  % bookmarks=true,            % PDF will contain an index on the RHS
  urlcolor=darkred,
  citecolor=darkblue,
  linkcolor=darkred
}
% The following defines a footnote without a marker
\newcommand\blfootnote[1]{%
  \begingroup
  \renewcommand\thefootnote{}\footnote{#1}%
  \addtocounter{footnote}{-1}%
  \endgroup
}

% -------- Customize page numbering ----------------------------
\usepackage{fancyhdr}
\usepackage{lastpage}
 \pagestyle{fancy}
\fancyhf{} % get rid of header and footer
 \cfoot{ \footnotesize{ \thepage } }

% \rhead{\small{\textit{Jianqi Huang}}}
% \lhead{\small{\textit{Heterogeneous Agents'n OLG model}}}
\renewcommand{\headrulewidth}{0pt} % turn off header line
% --------------------------------------------------------------

\begin{document}
\paragraph{Overview} This two papers \citet{hopenhayn1992entry} and \citet{hopenhayn1993job} are important papers in firm dynamics. 
\begin{itemize}
  \item DRS production function. 
  \item Perfect competition in product and labor market
  \item No aggregate shocks
  \item Idiosyncratic productivity shocks
  \item Entry-exit dynamics: Fixed cost to \textit{entry} and fixed cost to \textit{operate} firm each period.
  \item Introduce adjustment costs $\Rightarrow$ induce misallocation of resource across heterogeneous producers. 
\end{itemize}
\paragraph{Model}Preferences: $$ \max_{C_t,N_t}\sum_{t=0}^\infty \beta^t \ln C_t + A N_t\quad\text{s.t.} \quad p_t C_t = w_t N_t + \Pi_t + T_t $$ Note that if we plug $C_t$ into objective function, we will transfer the problem to a static problem. 
Production function: \begin{equation}
  p_t z f(n_t) - n_t - p_t c_f - g(n_t,n_{t-1}),\quad z \sim F(z)
\end{equation}
where $p_t$ is the price of output, $f(n_t)$ is the production function, $n_t$ is the number of workers, $c_f$ is the fixed cost to operate the firm, and $g(n_t,n_{t-1}) = \tau[n_{t-1}-n_{t}]^+$ is the adjustment cost. Note that $n_{t-1}$ is the state variable. 
% The productivity $z$ is drawn from a distribution $F(z)$ with support $[z_{\min},z_{\max}]$.

The timing of decision: exit: receive $-g(0,n_{t-1})$ this period zero in all future periods. Entry: make employment decision $n_t$ and receive $p_t z f(n_t)-n_t -g(n_t,n_{t-1})-p_t c_f$. 

Value function of incumbent firm:
\begin{equation}
  \begin{aligned}
  V(z,n) &= \max_{n'\geq 0} \bigg\{p z f(n') - n'  - g(n',n)- p c_f \\ &+ \beta \max[-g(0,n'),\int V(z',n')d F(z'\mid z)] \bigg\}
  \end{aligned}
\end{equation}
and the policy functions are $n' = n^d (z,n;p)$ and $\chi(z,n;p)\in \{0,1\}$. 

\paragraph{Entrants and Free entry condition}
Potential entrants are ex-ante identical. An entrants firm must pay the entry cost $c_e>0$ to set-up the plant and draw $z\sim G(z)$ starting producing next period with $n_{t-1}=0$. $M\geq 0$ mass of entrants. 
\textit{Free entry condition}: $$\beta \int V(z,0;p)\mathrm{d}G(z)\leq c_e$$ with strict equality if $M>0$ entrants.

\paragraph{Equilibrium}
Consider the stationary equilibrium: $\mu(z,n)$ denotes the distribution of firms across the state. 
\begin{equation} \mu_{t+1}(z',n') =\underbrace{ \int F(z'\mid z)(1-\chi(z,n))\mathbf{1}_{[n'=n^d(z,n)]} \mathrm{d}\mu_t (z,n)}_{\text{Incumbents don't exit}} + \underbrace{M_{t+1}G(z')\mathbf{1}_{[n'=0]}}_{\text{Entrants}} \end{equation}
% The transition function is $ F(z',n'\mid z,n) = F$ 
In stationary distribution, we have $\mu_{t+1}= \mu_t = \mu$. Again, linearity implies that $\mu(p,M) = M\times \mu(p,1)$. 


\paragraph{Aggregation}
Aggregate production and labor demand: 
$$ Y(p,M) = \int (z f(n^d(z,n;p))-c_f)\mathrm{d}\mu,\quad \text{and}\quad N^d(p,M) = \int n^d(z,n;p)\mathrm{d}\mu + M c_e $$
Expected firing tax revenue for a firm with state $(z,n)$ is 
\begin{equation}
  r(z,n;p ) = [1-\chi(z,n;p)]E_{z'\mid z}[g(n^d(z',n'(z,n)),n^d(z,n))]+\chi(z,n)g(0,n')
\end{equation}
and aggregate tax revenue $T(p,M) = \int r(z,n;p)\mathrm{d}\mu$. 

Aggregate profits: \begin{equation}
  \Pi (p,M) = p Y - N^d - T = \int \pi (z,n;p)d\mu- M c_e
\end{equation}



\paragraph{Solving Equilibrium}
Guess price $p^*$, and solve for the DPP of the incumbents. Check if $p^*$ satisfies the free entry $\beta \int V(z,0;p)\mathrm{d}G(z)= c_e$. If not, update $p^*$ and repeat. 

Given $p^*$, and the policy functions, assume $M = 1$ and solve for the stationary distribution $\mu(p^*,1)$. 
% Labor is a state variable, solving for the invariant distribution is harder. 
Using either the goods market or labor market clearing conditions\footnote{Actually, using the market clearing conditions is to connect firms and household's problems} to solve for $M$: 
$$ Y(p^*,M) = M\int (zf(n^d(z,n;p))-c_f )d\mu(p^*,1)= C(p^*) $$
Then, calculate the aggregate variables.

\paragraph{Implications} The adjustment cost implies that adjustment is \textcolor{blue}{lumpy}. 
If the adjustment cost is quadratic, firms will adjust slowly (no inaction region).
The linear adjustment cost induces the inaction region. 

Adjustment costs are less important if shocks are very persistent. 

The tax also prevents inefficient firms from exiting.

Note that misallocation here is induced by an aggregate friction. In more sophisticated models, the misallocation can be firm-specific.


\section*{Continuous-time version}
The problem of incumbent firm is
\begin{equation}
  \begin{aligned}
    V(z) &= \max_{\{n_t\}_{t\geq 0},\tau} \left\{\mathbb{E}_0 \int_0^\tau e^{-\rho t}(p f(z,n)-w n_t -c_f)\mathrm{d}t+e^{-\rho \tau}V^*\right\}
    \\ \mathrm{d}z_t &= \mu_z \mathrm{d}t + \sigma_z \mathrm{d}W_t,\quad z_0 = z.
  \end{aligned}
\end{equation}
We can rewrite the problem using HJBVI equation and KFE.
\begin{equation}
  \min \left\{\rho V(z)- V'(z)\mu(z)-\frac{1}{2}V''(z)\sigma^2(z)-\pi(z),V(z)-V^*\right\} = 0 ,\quad \text{all } z\in (0,1)
\end{equation}
\begin{equation}
  \partial_t g(z,t) = \partial_z(\mu(z)g(z,t)) + \frac{1}{2}\partial_{zz}(\sigma^2(z)g(z,t))+m(t)\psi(z), \quad \text{ all } z\in \mathcal{Z}
\end{equation}
where $\mathcal{Z}=[x,1]$ denotes the set of productivities such that firms remain in the industry. $m(t)$ denotes the exit rate (assume a constant mass of firms) which pinned down by the requirement that the total mass of firms is constant $\int_0^1 g(z,t)\mathrm{d}z = 1$. 

If there is a simple threshold rule for exit, i.e. exit whenever $z$ falls below $x$, then \begin{equation}
  m(t) = \frac{1}{2}\partial_z (\sigma^2 (x) g(x,t))
\end{equation}
Using the fact that $\int_x^1 g(z,t)\mathrm{d}z=1\Rightarrow \int_x^1 \partial_t g(z,t) = 0$.  
to get that 
\begin{equation}
  \begin{aligned}
     0 & = -\int_x^1 \partial_z (\mu(z)g(z,t))\mathrm{d}z + \frac{1}{2}\int_x^1 \partial_{zz}(\sigma^2 (z)g(z,t))\mathrm{d}z + m(t)\int_x^1 \psi(z)\mathrm{d}z \\
      &= \mu(x)g(x,t) - \frac{1}{2}\partial_z (\sigma^2 (x) g(x,t))
  \end{aligned}
\end{equation}
Further, using the diffusion process $\mathcal{A}V = \mu(z)\partial_z V+\frac{1}{2}\partial_{zz}V$ and $\mathcal{A}^*$ its adjoint $\mathcal{A}^* g = -\partial_z (\mu(z)g)+\frac{1}{2}\partial_{zz}(\sigma^2_{zz}g)$ so that the KFE is $\partial_t g = \mathcal{A}^* g + m(t)\psi(z)$. We then have \begin{equation}
  m(t) = \int_{\mathcal{Z}}(\mathcal{A}^* g)(z,t)\mathrm{d}z
\end{equation}
We can characterize the equilibrium by the following system of equations: 
\begin{equation}
  \begin{aligned}
  0 & =\min \left\{\rho V(z)-V^{\prime}(z) \mu(z)-\frac{1}{2} V^{\prime \prime}(z) \sigma^2(z)-\pi(z), V(z)-V^*\right\}, \quad \text { all } z \in(0,1) \\
  0 & =-(\mu(z) g(z))^{\prime}+\frac{1}{2}\left(\sigma^2(z) g(z)\right)^{\prime \prime}+m \psi(z), \quad \text { all } z \in \mathcal{Z} \\
  m & =-\int_{\mathcal{Z}}\left(\mathcal{A}^* g\right)(z) d z \\
  p & =D(Q), \quad w=W(N), \quad Q=\int_{\mathcal{Z}} q(z) g(z) d z, \quad N=\int_{\mathcal{Z}} n(z) g(z) \mathrm{d} z
  \end{aligned}
\end{equation}


\paragraph{Numerical Solution} relative to discrete time version, it requires that 



\bibliographystyle{aer}
\bibliography{ref.bib}


\end{document}